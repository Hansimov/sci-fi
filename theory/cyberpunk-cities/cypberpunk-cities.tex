\documentclass{beamer}

\usepackage[UTF8]{ctex}

\newfontfamily{\ttconsolas}{Consolas}
\setbeamerfont{main text}{size=\scriptsize, series=\ttconsolas}
% \setbeamerfont{main text}{size=\footnotesize} %, series=\bfseries}
% \setbeamerfont{main text}{size=\scriptsize} %, series=\bfseries}
\AtBeginDocument{\usebeamerfont{main text}}


\title{赛博朋克城市的常见要素}
\author{Hansimov}
\institute{2001神经漫游指南}
\date{2018.06}

\begin{document}

\begin{frame}
\titlepage
\end{frame}

% 动画:
%   攻壳机动队(多部)、阿基拉、苹果核战记(多部)、大都会(2001)
% 电影:
%   银翼杀手、黑客帝国(多部)、大都会(1927)

% Classic cyberpunk characters were marginalized, alienated loners who lived on the edge of society in generally dystopic futures where daily life was impacted by rapid technological change, an ubiquitous datasphere of computerized information, and invasive modification of the human body.
% — Lawrence Person

% 每一期主题都应当有一个主线,一切讨论都应当围绕主线。

% 这一期的主题是:

\begin{frame}

著名赛博朋克电影中的城市场景:

\begin{itemize}
    \item 银翼杀手(1982):00:03:06 -- 00:04:20, 00:07:25 -- 00:08:28
    \item 攻壳机动队(1995):00:33:03 -- 00:34:40 -- 00:36:20
    \item 攻壳机动队(2004):00:00:20 -- 00:01:00  ,  00:49:39 -- 00:52:29
    \item 大都会(2001):00:01:10 -- 00:03:17, 00:10:23 -- 00:13:18, 00:14:13 -- 00:14:40, 00:30:38 -- 00:32:58 | 00:34:50 -- 00:35:10(激光), 00:48:54 -- 00:50:05(贫民窟), 01:34:41 -- 01:38:56(毁灭)
\end{itemize}

\end{frame}


\end{document}