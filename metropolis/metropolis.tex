\documentclass[tikz, border=5pt]{standalone}

\usepackage{tikz}
\usetikzlibrary{arrows.meta,arrows}
% \usetikzlibrary{snakes}
\usetikzlibrary{calc}

\usepackage{varwidth}
\usepackage{calc}

% \usepackage[scheme=plain]{ctex}
\usepackage[UTF8]{ctex}
\setCJKmainfont{Microsoft YaHei}
\setmainfont{Microsoft YaHei}
% \CJKsetecglue{\hskip0.05em plus0.05em minus 0.05em}

\newcommand{\fs}[1]{\fontsize{#1 pt}{0pt}\selectfont}

% vertical vtick
% \newcommand{\vtick}[4]{
% % #1: tick id
% % #2: position
% % #3: name
% % #4: notes
%     \node [] (#1) at (#2, 0) {};
%     \draw [thick] (#1.center) -- +(0, -5pt);
%     \draw [thick] (#1.center) -- +(0,  5pt);
%     \node [anchor=south] at ([yshift= 1em] #1.north) {\begin{varwidth}{20em} #3 \end{varwidth}};
%     \node [anchor=north] at ([yshift=-1em] #1.south) {\begin{varwidth}{20em} #4 \end{varwidth}};
% }

\edef\prevpar{root}
\edef\thispar{root}
\edef\parwidth{30em}
\edef\leftgap{1em}
\edef\righgap{1em}
\edef\pargap{1.5ex}
\edef\htickwidth{10pt}

% \edef\xwidthmax{20 em}
% \newcommand{\getwidthof}[1]{\newlength{\tmplen} \settowidth{\tmplen}{#1} \the\tmplen}

% ygap = textheight * (int(xwidth/xwidthmax)+1)
% ypos = ypos + ygap

% horizontal tick
\newcommand{\htick}[2]{%
% #1: name
% #2: notes
    \edef\prevpar{\thispar}
    \edef\thispar{#1}

    \node [anchor=north west, align=left, draw=blue, yshift=-\pargap] (\thispar) at (\prevpar.south west) {\begin{varwidth}{\parwidth} #2 \end{varwidth}};

    \node [anchor=center] (\thispar0) at ([xshift=-\righgap] \thispar.west) {}; % This is just a mark, without any content

    \node [anchor=east, align=right] at ([xshift=-\leftgap] \thispar0.west) {\begin{varwidth}{10em} {\fs{12} \textbf{#1}} \end{varwidth}};

    \draw [line width=3pt] ([xshift=-\htickwidth/2]\thispar0.center) -- +(\htickwidth, 0);
}

\newcommand{\ps}{\\[1ex]}

\begin{document}

\begin{tikzpicture}[x=1em, y=1em]

\node [anchor=center, align=center] (title) {\fs{15}\textbf{大都会(2001)}};
\node [anchor=west] (root) at ([xshift=\righgap]title.south) {};

% \htick{开场}{
%     % 尽可能短,控制在15秒以内。\\ 可供选择的素材是开头的演讲或者庆祝盛典。
% };
% \htick{作品背景}{
%     % 改编、放映时间、相关作者、影响力
%     《大都会》(Metoroporisu)这部动画电影于2001年在日本上映,改编自手冢治虫1949年出版的同名漫画。
%     其实还有一部电影也叫《大都会》,于1927年在德国柏林首映,导演是弗里茨·朗(Fritz Lang)。
%     我最初接触到《大都会》(2001)这部动画电影,完全是因为偶然。
%     \ps
%     中学时我迷上了科幻作品,也因此久闻1927年那部科幻电影《大都会》的盛名,早就想一睹风采。终于等到高中的某个神奇的日子,那天电信赠送了我几百兆的流量,于是我一咬牙,当晚从百度视频上下载了一部名叫《大都会》的电影。
%     然而打开之后,我发现下载完成的不是那部我心心念念的传说中的德国表现主义科幻默片经典之作的《大都会》,而是一个不知道什么来头的莫名其妙的“动画片”。我大失所望、万念俱灰,而赠送的流量已经消耗殆尽、挥霍一空。
%     然而自己花流量下的动画片,跪着也要看完。
%     就这样,我一不小心踏入了日本动画电影的世界,也第一次接触到了优秀的科幻动画作品。
%     \ps
%     虽然有人说,这部动画电影兼具1949年漫画和1927年电影的风格,但是漫画本身,和1927年电影的直接关联却并不多。
%     至少手冢治虫本人,在漫画《大都会》的“后记”中这样写道:
%     \ps
%     “这个人造人的构想原本是来自对战前德国的电影名作《大都会》中的女性机器人印象,但话虽如此,我在作画之前并未观赏过这部影片,也不晓得内容如何。这只不过是我从战争时的《电影旬报》当中所刊载的一页剧照,对女机器人诞生的场景有感而发而铭记在心的一种提示使然。《大都会》这部电影受到观众很大的回响,而我虽然使用和他相同的题名,事实上此书和这部电影并无特别的关联。”
% };
% \htick{情节/角色}{
%     %% 以主角的视角?分成不同的阵营?
%     回到这部2001年的动画电影上来。
%     高中时,我还不认识片头那些如雷贯耳的名字,既不知道手冢治虫,也不知道大友克洋。
%     % 也对ACG(Anime(日本动画)、Comics(漫画)、 Games(游戏))文化一无所知。
%     实话说,当时这部动画电影看得我是稀里糊涂、一头雾水。
%     \ps
%     以我当时的视角和水平来看:
%     \ps
%     全片节奏缓慢拖沓,大部分时间似乎都是人物赶路和场景切换,让人有种在看旅游宣传片的感觉,浑然不知主角正在面临的危险和即将采取的行动;
%     人物行为也莫名其妙,掌权者不干正事,醉心于将人造机器人送上王座,而叔叔对走失的侄子毫不在意,不仅内心毫无波动,甚至还有工夫吃点霸王餐;
%     主题上更是老调重弹,无非是人与机器人之间的感情与矛盾,统治者互相勾心斗角、尔虞我诈,底层人民饱受压迫、生活水深火热,起义者热血勇敢、可惜智商欠费,好不容易有个“我是谁”这样的哲学问题若隐若现,然而不见影片有任何深入的追问和探讨。
%     \ps
%     总而言之,对当时的我来说,在电影前十分之九的时间里,我都在因为虚度的年华而悔恨,更因为浪费的流量而愤懑。
%     所以,如果你习惯了商业科幻大片那样简单流畅的叙事方式,希望能在闲暇时光放松自己的话,那么《大都会》(2001)这部动画电影一定不是你想要的类型。
% };
% \htick{独特之处}{
%     % 讲述推荐理由,否则视频就沦为通俗意义上的泛泛而谈。
%     然而,如果说这部动画电影的故事性乏善可陈的话,那么在精良的画面和配乐,则让人将那些缺点忘个干净。
%     \ps
%     我相信,很多看过这部动画电影的观众,都像我一样,虽然不知电影所云,但是总会在准备关掉视频的一刹那,心生一丝犹豫。
%     精致到不真实的场景、简笔画式的人物造型、慵懒自由的爵士乐,都同技术高度发达的大都会格格不入,可是这反倒产生了一种奇怪的美感。
%     \ps
%     % 蒸汽朋克/真空管朋克、机械
%     高耸入云的通天巨塔,漫天绽放的璀璨烟花,五彩斑斓的城市街道,阴冷潮湿的地下迷宫,复古诡异的实验基地,精密复杂的巨型武器,
%     % 片尾毁灭
% };

\draw [-Stealth, thick] ([yshift=-1ex]title.south) -- ([xshift=-\righgap, yshift=-1ex]\thispar.south west);

\end{tikzpicture}


\end{document}